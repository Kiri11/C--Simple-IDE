\documentclass{article}

% Языковые пакеты
\usepackage{fontspec} 
\usepackage{xunicode}
\usepackage{xltxtra}

% Используемый шрифт - Times New Roman
\setmainfont[Mapping=tex-text]{Times New Roman}
% Используемые языки
\usepackage{polyglossia}

\setdefaultlanguage[spelling=modern]{russian}
\setotherlanguages{english}
    \AtBeginDocument{%
      \def\postsection{\quad}%
      \def\postsubsection{\quad}%
      \def\postsubsubsection{\quad}%
      \def\postparagraph{\quad}%
      \def\postsubparagraph{\quad}%
    }

\newfontfamily{\cyrillicfonttt}{Courier New}
% Поля страницы
\usepackage{geometry}
\geometry{left=3cm}
\geometry{right=2cm}
\geometry{top=2cm}
\geometry{bottom=2cm}
% Использовать отступы  вначале абзаца
\usepackage{indentfirst}
\usepackage{fancyhdr}
\pagestyle{fancy}
\fancyhead{}
\renewcommand{\headrulewidth}{0pt} % убираем линию вверху каждой страницы(устанавливаем ширину ее в 0)
\lhead{}
\chead{}
\rhead{}
\lfoot{}
\cfoot{}
\rfoot{\thepage}

\usepackage{listings}
\usepackage{color}

\definecolor{mygreen}{rgb}{0,0.6,0}
\definecolor{mygray}{rgb}{0.5,0.5,0.5}
\definecolor{mymauve}{rgb}{0.58,0,0.82}
\definecolor{light-gray}{gray}{0.95}
\definecolor{orange}{RGB}{255,127,0}

\lstset{ %
  backgroundcolor=\color{white},   % choose the background color; you must add \usepackage{color} or \usepackage{xcolor}
  basicstyle=\footnotesize\ttfamily,        % the size of the fonts that are used for the code
  breakatwhitespace=false,         % sets if automatic breaks should only happen at whitespace
  breaklines=true,                 % sets automatic line breaking
  captionpos=b,                    % sets the caption-position to bottom
  commentstyle=\color{mygreen},    % comment style
  deletekeywords={...},            % if you want to delete keywords from the given language
  escapeinside={\%*}{*)},          % if you want to add LaTeX within your code
  extendedchars=true,              % lets you use non-ASCII characters; for 8-bits encodings only, does not work with UTF-8
  frame=single,                    % adds a frame around the code
  keepspaces=true,                 % keeps spaces in text, useful for keeping indentation of code (possibly needs columns=flexible)
  keywordstyle=\color{blue},       % keyword style
  language=C++,                 % the language of the code
  morekeywords={*,...},            % if you want to add more keywords to the set
  numbers=left,                    % where to put the line-numbers; possible values are (none, left, right)
  numbersep=5pt,                   % how far the line-numbers are from the code
  numberstyle=\tiny\color{mygray}, % the style that is used for the line-numbers
  rulecolor=\color{black},         % if not set, the frame-color may be changed on line-breaks within not-black text (e.g. comments (green here))
  showspaces=false,                % show spaces everywhere adding particular underscores; it overrides 'showstringspaces'
  showstringspaces=false,          % underline spaces within strings only
  showtabs=false,                  % show tabs within strings adding particular underscores
  stepnumber=1,                    % the step between two line-numbers. If it's 1, each line will be numbered
  stringstyle=\color{mymauve},     % string literal style
  tabsize=3,                       % sets default tabsize to 3 spaces
  title=\lstname                   % show the filename of files included with \lstinputlisting; also try caption instead of title
}

\begin{document}
\begin{center}
\Huge Правила оформления исходных текстов.
\end{center}

\Large Отступы.
\normalsize
\vskip0.5cm
Величина отступа равняется \textbf{3} пробелам, причем символ табуляции заменяется соответствующим количеством пробельных символов.
\vskip0.5cm
\Large Выражения и скобки в выражениях.
\normalsize
\vskip0.5cm
Выражения и аргументы функций, заключенные между круглыми, угловыми, либо квадратными скобками, отделяются от скобок одним пробелом. Если используется пустая скобочная последовательность \textbf{()}, \textbf{\{\}}, \textbf{<>}, \textbf{[]}, тогда пробела между скобками не ставится. Между операторами в выражении также ставятся пробелы. После ключевых слов типа \texttt{if}, \texttt{for} и т.п., а также после имен классов у функций пробел перед круглыми скобками не ставится. Примеры:
\begin{lstlisting}
template< typename T >
void Func( T a )
{
   a.DoSomething();
}

template<>
void Func< int >( int a )
{
   a = ( 1 + 2 ) * 3;
}

for( int j = 0; j < n; ++j )
\end{lstlisting}
\vskip0.5cm

\Large Оператор <<?:>>.
\normalsize
\vskip0.5cm
При написании оператора <<?:>> каждая часть оператора должна быть отделена пробелами:
\begin{lstlisting}
5 == a ? 1 + b : *vec.begin();
\end{lstlisting}
\color{red}Недопустимо использование оператора <<?:>> в качестве замены условиям <<if>>.
\color{black}

\vskip0.5cm

\Large CV-квалификаторы типов переменных, указатели и ссылки.
\normalsize
\vskip0.5cm
Квалификаторы \texttt{\textbf{const}} и \texttt{\textbf{volatile}} идут после имени типа при объявлении переменных и обозначении типов возвращаемого значения для функции. Символы \texttt{\textbf{*}} и \texttt{\textbf{\&}} пробелом слева не отделяется от типа, а справа от имени переменной отделяется одним пробелом. Пример:
\begin{lstlisting}
char const* a; // указатель на char const
char* const a; // константный указатель на char

int F( std::vector< int > const& v )
{
   assert( !v.empty() );
   return v[ 0 ];
}
\end{lstlisting}
\vskip0.5cm

\Large Группировка условий в сравнениях.
\normalsize
\vskip0.5cm
Условия проверки на равенство и неравенство(операторы \texttt{==} и \texttt{!=}) выполняются в стиле left-hand comparison, то есть по левую сторону от знака сравнения стоит rvalue, а по правую -- lvalue сущность. Пример:
\begin{lstlisting}
int exit_code = Procedure();

if( 0 == exit_code )
   DoSomething();
\end{lstlisting}

Неправильный вариант:
\lstset{ %
  backgroundcolor=\color{orange}}

\begin{lstlisting}
int exit_code = Procedure();

if( exit_code == 0 )
   DoSomething();
\end{lstlisting}
\lstset{ %
  backgroundcolor=\color{white}}

Несколько условий группируются так, чтобы знаки конкатенации условий \texttt{||}, \texttt{\&\&} находились на той же строке, что и первое условие из комбинации двух условий, сами условия, если их несколько, заключаются в скобки и выравниваются, чтобы они начинались строго друг под другом. Пример:
\begin{lstlisting}
if( ( 1 == a ) &&
    ( invalid_size != vec.size() ) &&
    ( std::string( "blah" ) == my_str ) )
{
   DoSomething();
}

if( ( ( 1 == a ) &&
      ( invalid_size != vec.size() ) ) ||
    ( something >= 12 ) )
{
   DoSomething();
}  
\end{lstlisting}

\vskip0.5cm
\Large Фигурные скобки.
\normalsize
\vskip0.5cm
Открывающая фигурная скобка ставится на новой строке без отступа. Текст, следующий после скобки, начинается со следующей строки с отступом. Закрывающая фигурная скобка ставится ровно под открывающей. Пример:
\begin{lstlisting}
if( 1 == a )
{
   a = 2;
}

do
{
   DoSomething();
}
while( true == condition );
\end{lstlisting}

\vskip0.5cm
\Large Правила наименования переменных.
\normalsize
\vskip0.5cm
Недопустимо использование ведущих символов нижнего подчёркивания в названиях переменных. Транслитерация имен переменных(то есть написание русских слов английскими буквами) тоже недопустима.
Переменные-члены класса начинаются с префикса <<\texttt{m}>>(например: \texttt{mIntValue}), а далее имя переменной продолжается в camel case. 
Имена аргументов функции также следуют в camel case, только первое слово начинается с маленькой буквы(например: \texttt{myArgument}). 
Имена локальных переменных в функциях именуются по традиционным правилам именования в C++, когда все слова пишутся с маленькой буквы, а отдельные слова разделяются символами нижнего подчёркивания(пример: \texttt{my\_var}). Имена классов, функций, шаблонных аргументов также подчиняются соглашению camel case, в данном случае все слова, включая первое, пишутся с большой буквы(пример: \texttt{class MyClassFactory : public AbstractFactory}).
Функции, возвращающие значения опций, состояний, размеров должны именоваться без префикса «Get». Например: int Size(),  int Options().
Функции, устанавливающие значения опций, состояний, размеров должны именоваться c префиксом «Set». Например: SetSize(int),  SetOptions(int).
Функции, проверяющие какое состояние и возвращающие bool,  должны именоваться с префиксом «Is». Например: bool IsVisible().


Глобальные переменные, все функции, методы классов, типы данных, классы – все слова в имени должны начинаться с буквы в верхнем регистре, остальные буквы в нижнем регистре, подчёркивание не используется. Например: MainDB, Buf, DataView, GetPos.


Константы и перечисления (enum) – для каждой группы констант заводится префикс – несколько букв в нижнем регистре, например, для перечисления «ResourceTypes» префикс будет «rt». Имена всех констант данной группы должны начинаться с такого префикса, все остальные буквы имени должны быть в верхнем регистре, слова в имени разделяются подчёркиванием. Например: rtMENU\_BAR, ftTEXT, msgSET\_TEXT.


Метки оператора goto, а также имена макросов и константы пишутся в верхнем регистре, слова разделяются подчёркиванием. Например: DONT\_USE\_GOTO, CONST\_STRING, MY\_MACRO.

\vskip0.5cm
\Large Правила описания функций.
\normalsize
\vskip0.5cm
При описании функций и методов классов не допускается опускать имена параметров, если не очевидно, что этот параметр означает. Например, так писать нельзя:
\begin{lstlisting}
int SetParams(int,int,int); 
\end{lstlisting}
Нужно писать так:
\begin{lstlisting}
int SetParams( int x, int y, int size );
\end{lstlisting}
А вот так писать можно, так как из наименования функции и типа параметра ясно, что должна делать данная функция:
\begin{lstlisting}
int Delete( Entry& ); 
\end{lstlisting}

\vskip0.5cm
\Large Правила описания классов.
\normalsize
\vskip0.5cm
В описании класса сначала должен идти раздел public, далее раздел protected, а потом private. Внутри каждого раздела должны быть сначала описаны методы, а потом поля класса. 
Перед словами «protected» и «private» должна быть оставлена пустая строка. Пустая строка должна быть и между полями и методами класса. 
Имена полей и методов класса выравниваются по левому краю, типы полей и типы возвращаемых методами значений выравниваются по правому краю.
Перечисление «друзей» класса идёт самым последним перед закрывающей скобкой.
Если класс переопределяет наследуемый виртуальный метод, то перед этим методом обязательно нужно указать «virtual».
Пример описания класса:
\begin{lstlisting}
class MyItem : public AbstractItem
{
   typedef AbstractItem MyParent;
   
   public:
   
   enum ItemType
   {
      itSIMPLE,
      itCOMPOSITE
   };
   
   
                       MyItem( ItemType );
   virtual            ~MyItem();
   
    std::string const& Name() const;
                  void SetName( std::string const& name );
   
              ItemType Type() const;
                  void SetType( ItemType type );
   
   protected:
   
   virtual        void DoSave() const;
   virtual        void DoLoad();
   
   virtual std::string HandleEvent( Event evt )
                       {
                          // некоторая своя обработка
                          return MyParent::HandleEvent( evt );
                       }
   
   private:
   
           std::string mName;
   std::vector< Opts > mOptions;
              ItemType mType;
};
\end{lstlisting}

\vskip0.5cm
\Large Правила написания комментариев.
\normalsize
\vskip0.5cm
Комментирование классов, методов и различных флагов крайне желательно. Комментарии к классам и функциям даются в нотации doxygen'а:
\begin{lstlisting}
/**
 * \author Вася Пупкин
 * \brief Функция, вычисляющая длину окружности радиуса radius
 * \param radius радиус окружности
 * \details если radius < 0, будет сгенерировано исключение типа InvalidRadiusException
 */
double CircleLength( double radius );
\end{lstlisting}

\vskip0.5cm
\Large Прочие рекомендации по стилю программирования.
\normalsize
\vskip0.5cm
Крайне не рекомендуется использование оператора \texttt{goto}! Константы не стоит определять с помощью макросов \texttt{define}, и вообще их использование в целом нежелательно и его стоит избегать. Магические числа в коде не допускаются, нужно заводить именованные типизированные константы, помеченные квалификатором \texttt{const}.

Если функция принимает указатель или ссылку на нечто, что не должно этой функцией изменяться, то не забывайте указывать «const». Особенно это касается параметров «char*». 

Если метод класса не изменяет поля класса, то указывайте «const» после описания такого метода. Например: int Count() const;. 

В большинстве случаев экземпляры классов в функцию нужно передавать по ссылке.  
\end{document}